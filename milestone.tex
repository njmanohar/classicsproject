
\documentclass[letterpaper,twocolumn,10pt]{article}
\usepackage{usenix,epsfig,endnotes}
\usepackage{mathtools}
\usepackage{amssymb}
\usepackage{amsfonts}
\usepackage{amsmath}
\usepackage{amsthm}
\usepackage{ifthen}
\usepackage{enumerate}
\usepackage{latexsym}
\usepackage{graphicx}
\usepackage{textcomp}
\usepackage{color}
\usepackage[lofdepth,lotdepth]{subfig}
\usepackage{tablefootnote}
\usepackage{amssymb}
\usepackage{pifont}
\usepackage{tikz}
\usepackage{hyperref}

\begin{document}

%don't want date printed
\date{}

%make title bold and 14 pt font (Latex default is non-bold, 16 pt)
\title{\Large \bf Project Milestone}

%for single author (just remove % characters)
\author{
{\rm Nathan Manohar}\\
\and
{\rm Ben Fisch}\\
% copy the following lines to add more authors
% \and
% {\rm Name}\\
%Name Institution
} % end author

\maketitle

% Use the following at camera-ready time to suppress page numbers.
% Comment it out when you first submit the paper for review.
\thispagestyle{empty}


%\subsection*{Abstract}
%Your Abstract Text Goes Here.  Just a few facts.
%Whet our appetites.

\section{Introduction and Problem Overview}

The problem of factoring $p^2 q$ is of considerable interest in cryptography. Various encryption schemes, such as the EPOC cryptosystem~\cite{Okamoto}, are based on the assumption that this problem is hard to solve. A natural question to ask is whether factoring $p^2q$ is as hard as factoring a general RSA modulus $N = pq$. In particular, is there a way to exploit the fact that $N$ is of the form $p^2q$ that could lead to an improvement in factoring moduli of this form? One observation is that given $N = p^2q$, it is easy to learn if $q$ is a quadratic residue modulo some prime $l$. To see this, we first note that the Jacobi symbol $\left(\frac{l}{N}\right)$ is computable in polynomial time via repeated application of the law of quadratic reciprocity. Additionally, since the Jacobi symbol is a multiplicative function, it follows that
\[
\left(\frac{l}{N}\right) = \left(\frac{l}{p^2q}\right) = \left(\frac{l}{p^2}\right) \left(\frac{l}{q}\right) = \left(\frac{l}{p}\right)^2 \left(\frac{l}{q}\right).
\]

Since $\text{gcd}(l,p) = 1$ (for $l \ne p$), it follows that the Legendre symbol $\left(\frac{l}{p}\right) = \pm 1$ and therefore
\[
\left(\frac{l}{N}\right) = \left(\frac{l}{q}\right).
\]
By the law of quadratic reciprocity, it follows that
\[
\left(\frac{l}{q}\right) \left(\frac{q}{l}\right) = (-1)^{\frac{l-1}{2} \frac{q-1}{2}}, 
\]
and so $\left(\frac{q}{l}\right)$ can be computed in polynomial time.

\section{This is Another Section}


\section{This Section has SubSections}





{\footnotesize \bibliographystyle{acm}
\bibliography{milestone}}


%\theendnotes

\end{document}
