
\documentclass[letterpaper,twocolumn,10pt]{article}
\usepackage{usenix,epsfig,endnotes}
\usepackage{mathtools}
\usepackage{amssymb}
\usepackage{amsfonts}
\usepackage{amsmath}
\usepackage{amsthm}
\usepackage{ifthen}
\usepackage{enumerate}
\usepackage{latexsym}
\usepackage{graphicx}
\usepackage{textcomp}
\usepackage{color}
\usepackage[lofdepth,lotdepth]{subfig}
\usepackage{tablefootnote}
\usepackage{amssymb}
\usepackage{pifont}
\usepackage{tikz}
\usepackage{hyperref}

\begin{document}

%don't want date printed
\date{}

%make title bold and 14 pt font (Latex default is non-bold, 16 pt)
\title{\Large \bf Project Milestone}

%for single author (just remove % characters)
\author{
{\rm Nathan Manohar}\\
\and
{\rm Ben Fisch}\\
% copy the following lines to add more authors
% \and
% {\rm Name}\\
%Name Institution
} % end author

\maketitle

% Use the following at camera-ready time to suppress page numbers.
% Comment it out when you first submit the paper for review.
\thispagestyle{empty}


%\subsection*{Abstract}
%Your Abstract Text Goes Here.  Just a few facts.
%Whet our appetites.

\section{Introduction and Problem Overview}

The problem of factoring $p^2 q$ is of considerable interest in cryptography. Various encryption schemes, such as the EPOC cryptosystem~\cite{Okamoto}, are based on the assumption that this problem is hard to solve. A natural question to ask is whether factoring $p^2q$ is as hard as factoring a general RSA modulus $N = pq$. In particular, is there a way to exploit the fact that $N$ is of the form $p^2q$ that could lead to an improvement in factoring moduli of this form? One observation is that given $N = p^2q$, it is easy to learn if $q$ is a quadratic residue modulo some prime $l$. To see this, we first note that the Jacobi symbol $\left(\frac{l}{N}\right)$ is computable in polynomial time via repeated application of the law of quadratic reciprocity. Additionally, since the Jacobi symbol is a multiplicative function, it follows that
\[
\left(\frac{l}{N}\right) = \left(\frac{l}{p^2q}\right) = \left(\frac{l}{p^2}\right) \left(\frac{l}{q}\right) = \left(\frac{l}{p}\right)^2 \left(\frac{l}{q}\right).
\]

Since $\text{gcd}(l,p) = 1$ (for $l \ne p$), it follows that the Legendre symbol $\left(\frac{l}{p}\right) = \pm 1$ and therefore
\[
\left(\frac{l}{N}\right) = \left(\frac{l}{q}\right).
\]
By the law of quadratic reciprocity, it follows that
\[
\left(\frac{l}{q}\right) \left(\frac{q}{l}\right) = (-1)^{\frac{l-1}{2} \frac{q-1}{2}}, 
\]
and so $\left(\frac{q}{l}\right)$ can be computed in polynomial time.

Using this information, we can construct tables $T_{l_i}$ for primes $l_i$ that list the possible values of $q \bmod l_i$. Since there are $\frac{l_i - 1}{2}$ quadratic residues and nonresidues modulo $l_i$, the size of $T_{l_i}$ will be $\frac{l_i - 1}{2}$. The question of factoring $p^2 q$ has now been reduced to the question of whether $q$ can be efficiently reconstructed given information about whether or not $q$ is a quadratic residue modulo primes for a fixed sequence of primes. 

\section{One Approach}

One approach to factoring $p^2 q$ is to attempt to construct a polynomial that must have $q$ as a small root and then determine this root using Coppersmith's method~\cite{Coppersmith}. Observe that for each prime $l_i$ for which we have a corresponding table $T_{l_i}$, we can construct the polynomial
\[
f_{l_i}(x) = \prod_{a \in T_{l_i}} (x - a) \bmod l_i.
\]

Then, using the Chinese remainder theorem, we can construct the polynomial $f \bmod \prod_i l_i$ obtained by applying to the Chinese remainder theorem term-wise to the coefficients of each of the $f_{l_i}$'s. Note that since $q$ is a root of all the $f_{l_i}$'s, it follows that $q$ is a root of $f$. However, $q$ is not a sufficiently small root of $f$ for Coppersmith's algorithm to determine it. In particular, if $f$ is a polynomial of degree $d$ modulo $L$, Coppersmith's algorithm will only find roots that are $< L^{1/d}$. However, if the $l_i$'s are the first $n$ primes, then $\prod_i l_i \approx e^n$. Since we must have $\prod_i l_i > q$, it follows that we need $n > \ln q$. Additionally, we note that the degree of $f$ will be the size of the largest table, which will be $\frac{l_n - 1}{2}$. Since the size of the largest prime is $\approx n \ln n$, it follows that $L^{1/d}$ will be approximately
\[
e^{2n/n\ln n} = e^{2/\ln n} = e^{O(1/\log \log q)}  << q,
\]  
and so Coppersmith's method will not be able to determine $q$.

An immediate observation is that if the degree of $f$ was smaller (was $n^{1 - \varepsilon}$ instead of $\approx n \ln n$), then Coppersmith's method would be sufficient to determine $q$. In this case, we would have that $L^{1/d}$ is $\approx e^{n^{\varepsilon}} > q$ for $n = O(\log^{1/\varepsilon} q)$. However, there does not seem to be any way to reduce the degree using these tables alone. In particular, since the size of the tables is $O(n \ln n)$, $f$ must necessarily have degree $O(n \ln n)$ in order to capture all the possible values of $q$. If some extra information about $q$ unrelated to the quadratic residuosity of $q$ modulo primes could be gathered, this could potentially be leveraged to reduce the size of the tables. 

\section{Future Work}





{\footnotesize \bibliographystyle{acm}
\bibliography{milestone}}


%\theendnotes

\end{document}
